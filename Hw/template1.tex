\documentclass[12pt]{article}
\usepackage{amsfonts,amsmath,amssymb,graphicx,url}

% Old Stuff
%%\oddsidemargin=0.15in
%%\evensidemargin=0.15in
%%\topmargin=-.5in
%%\textheight=9in
%%\textwidth=6.25in

\setlength{\oddsidemargin}{.25in}
\setlength{\evensidemargin}{.25in}
\setlength{\textwidth}{6.25in}
\setlength{\topmargin}{-0.4in}
\setlength{\textheight}{8.5in}

\newcommand{\heading}[5]{
   \renewcommand{\thepage}{#1-\arabic{page}}
   \noindent
   \begin{center}
   \framebox{
      \vbox{
    \hbox to 6.2in { {\bf CS390 Computational Game Theory and Mechanism Design}
     	 \hfill #2 }
       \vspace{4mm}
       \hbox to 6.2in { {\Large \hfill #5  \hfill} }
       \vspace{2mm}
       \hbox to 6.2in { {\it #3 \hfill #4} }
      }
   }
   \end{center}
   \vspace*{4mm}
}

\newcommand{\handout}[3]{\heading{#1}{#2}{Huang Zen}{}{#3}}

\setlength{\parindent}{0in}
\setlength{\parskip}{0.1in}

\begin{document}
\handout{1}{June 30, 2013}{Problem Set 1}

\paragraph{Problem 1} (Collaborated with AAA, BBB, and CCC.)

Your solution goes here. $\ldots$. By Theorem X of \cite{OR94}, pp. 23, the desired result holds.

\bigskip

\paragraph{Problem 2} (Collaborated with YYY and ZZZ.)

Your solution goes here.

\bigskip

\paragraph{Problem 3} (No collaborator.)

Your solution goes here.

\bibliographystyle{agsm}

\begin{thebibliography}{99}

\bibitem{OR94}{M. J. Osborne and A. Rubinstein. {\em A course in game theory.} MIT Press, 1994.}

\bibitem{NRTV07}{N. Nisan, T. Roughgarden, E. Tardos, and V. Vazirani (eds). {\em Algorithmic game theory.} Cambridge University Press, 2007. (Available at \url{http://www.cambridge.org/journals/nisan/downloads/Nisan_Non-printable.pdf}.)}

\end{thebibliography}

\end{document}








